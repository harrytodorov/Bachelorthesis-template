\documentclass[12pt, 
               a4paper, 
               openright,
               abstracton]{scrreprt}

\usepackage[utf8]{inputenc}
\usepackage[T1]{fontenc}
\usepackage[ngerman]{babel}
\usepackage{graphicx}
\usepackage[printonlyused]{acronym}
\usepackage{setspace} \renewcommand{\baselinestretch}{1.5}          % use 1.5 line space (acc. DHBW Stuttgart Formalia)
\usepackage[top=25mm, left=25mm, right=30mm, bottom=25mm]{geometry} % define margins (acc. DHBW Stuttgart Formalia)
\usepackage[tocbib, bibnewpage]{apacite}
\usepackage{fancyhdr}
\usepackage[hyphens]{url}                                           % format URLs the right way

\newcommand{\titleLong}{Full title of the thesis}
\newcommand{\titleShort}{Short title of the thesis}
\newcommand{\company}{Company}

\pagestyle{fancy}                                                   % sets fancy page headers and footers
\fancyhf{}                                                          % removes default headers and footers
\lhead{\titleShort}
\cfoot{\nouppercase{\leftmark}}
\rfoot{\thepage}
\renewcommand{\headrulewidth}{1.5pt}
\renewcommand{\footrulewidth}{1pt}

\begin{document}
    % Title page
    \begin{titlepage}
    \includegraphics[height=40pt]{images/company_logo} % logo of your company
    \hfill
    \includegraphics[height=40pt]{images/dhbw.png} % DHBW logo
    
    \enlargethispage{25mm}
    \begin{center}
        \vspace*{12mm}  {\LARGE\bf \titleLong} \\
        \vspace*{12mm}  {\LARGE\bf Bachelorarbeit}\\
        \vspace*{12mm}  für die Prüfung zum\\
        \vspace*{3mm}   {\large Bachelor of Science}\\
        \vspace*{12mm}  des Studienganges Angewandte Informatik\\
        \vspace*{3mm}   an der Dualen Hochschule Baden-Württemberg Stuttgart\\
        \vspace*{12mm}  von\\
        \vspace*{3mm}   {\large Haralambi Todorov}\\
        \vspace*{12mm}  {\large\bf 05.09.2016}
    \end{center}
    \vfill
    \begin{spacing}{1.5}
        \begin{tabbing}
			% first line is just for the sake of formatting
			aaaaaaaaaaaaaaaaaaaaaaaaaaaaaaa\quad\= aaaaaaaaaaaaaaaaaaaaaaaaa\kill
            Bearbeitungszeitraum \> 12 Wochen\\
            Matrikelnummer, Kurs \> MatrikelNr., Kurs\\
            Ausbildungsfirma \> Company name and address\\
            Betreuer der Ausbildungsfirma \> Betreuer in Unternehmen\\
            Gutachter der Dualen Hochschule \> Betreuer an Uni\\
        \end{tabbing}
    \end{spacing}
\end{titlepage}
    
    % start counting of pages <roman style>
    \renewcommand{\thepage}{\Roman{page}}
	\setcounter{page}{1}

    % Solemn declaration
    \chapter*{Erklärung}
\addcontentsline{toc}{chapter}{Erklärung}

gemäß § 5 (3) der \glqq{}Studien- und Prüfungsordnung DHBW Technik\grqq{} vom 22. September 2011.
\\
Ich habe die vorliegende Arbeit selbstständig verfasst und keine anderen als die angegebenen Quellen und Hilfsmittel verwendet.
\vspace{4em}

\begin{tabular}{lp{2em}l}
	\hspace{3cm}   && \hspace{3cm} \\\cline{1-1}\cline{3-3}
	Ort, Datum     && Unterschrift
\end{tabular} 
    \newpage
    
    % Abstract German
\renewcommand{\abstractname}{Zusammenfassung}
\begin{abstract}
    % Abstract in German
\end{abstract}

\newpage

% Abstract English
\renewcommand{\abstractname}{Abstract}
\addcontentsline{toc}{chapter}{Abstract}
\begin{abstract}
    % Abstract in English
\end{abstract}
    \newpage
    
    \tableofcontents                                                % table of contents
    \newpage
    
    % start counting the pages <arabic style>
    \renewcommand{\thepage}{\arabic{page}}
    \setcounter{page}{1}

    \input{contents/01einleitung.tex}                               % Einleitung
    \input{contents/02standtechnik.tex}                             % Stand der Technik
    \input{contents/03anforderungsanalyse.tex}                      % Anforderungsanalyse
    \chapter{Erarbeitung eines Einführungskonzeptes}
                           % Erarbeitung eines Einführungskonzeptes
    \input{contents/05umsetzung.tex}                                % Umsetzung des Konzeptes
    \input{contents/06ausblick.tex}                                 % Ausblick
    
    \chapter*{Abkürzungsverzeichnis}
\addcontentsline{toc}{chapter}{Abkürzungsverzeichnis}
\begin{acronym}
    \acro{erp}[ERP]{Enterprise Resource Planning}
    \acrodefplural{banf}[BANF]{Bedarfsanforderungen}
\end{acronym}                                 % list of acronyms
    \listoffigures                                                  % list of figures
    \addcontentsline{toc}{chapter}{\listfigurename}
    
    \bibliographystyle{apacite}
    \bibliography{additional/quellen.bib}
\end{document}